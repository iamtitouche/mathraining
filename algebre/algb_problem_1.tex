 \documentclass[paper=a4, fontsize=11pt]{scrartcl} % A4 paper and 11pt font size

\usepackage[T1]{fontenc} % Use 8-bit encoding that has 256 glyphs
\usepackage{fourier} % Use the Adobe Utopia font for the document - comment this line to return to the LaTeX default
\usepackage[english]{babel} % English language/hyphenation
\usepackage{amsmath,amsfonts,amsthm,amssymb} % Math packages

\usepackage{lipsum} % Used for inserting dummy 'Lorem ipsum' text into the template

\usepackage{sectsty} % Allows customizing section commands
\allsectionsfont{\centering \normalfont\scshape} % Make all sections centered, the default font and small caps
\usepackage{multicol}
\usepackage{fancyhdr} % Custom headers and footers
\pagestyle{fancyplain} % Makes all pages in the document conform to the custom headers and footers
\fancyhead{} % No page header - if you want one, create it in the same way as the footers below
\fancyfoot[L]{} % Empty left footer
\fancyfoot[C]{} % Empty center footer
\fancyfoot[R]{\thepage} % Page numbering for right footer
\renewcommand{\headrulewidth}{0pt} % Remove header underlines
\renewcommand{\footrulewidth}{0pt} % Remove footer underlines
\setlength{\headheight}{13.6pt} % Customize the height of the header

\numberwithin{equation}{section} % Number equations within sections (i.e. 1.1, 1.2, 2.1, 2.2 instead of 1, 2, 3, 4)
\numberwithin{figure}{section} % Number figures within sections (i.e. 1.1, 1.2, 2.1, 2.2 instead of 1, 2, 3, 4)
\numberwithin{table}{section} % Number tables within sections (i.e. 1.1, 1.2, 2.1, 2.2 instead of 1, 2, 3, 4)

\setlength\parindent{0pt} % Removes all indentation from paragraphs - comment this line for an assignment with lots of text

%----------------------------------------------------------------------------------------
%	TITLE SECTION
%----------------------------------------------------------------------------------------

\newcommand{\horrule}[1]{\rule{\linewidth}{#1}} % Create horizontal rule command with 1 argument of height

\title{	
\normalfont \normalsize 
\textsc{} \\ [25pt] % Your university, school and/or department name(s)
\horrule{0.5pt} \\[0.4cm] % Thin top horizontal rule
\huge Problème n°1 - Algèbre\\ % The assignment title
\horrule{2pt} \\[0.5cm] % Thick bottom horizontal rule
}

\author{Sadoun Titouan} % Your name

\date{\normalsize\today} % Today's date or a custom date

\begin{document}

\maketitle % Print the title

%----------------------------------------------------------------------------------------
%	PROBLEM 1
%----------------------------------------------------------------------------------------

Cherchons par analyse-synthèse, les polynômes $p$ de $\mathbb{R}[X]$ tels que : $$\forall x \in \mathbb{R}, (x - 16)p(2x) = 16(x - 1)p(x) \text{     } (E)$$


Analyse : Soit $p \in \mathbb{R}[X]$ satisfaisant l'égalité $(E)$ pour tout réel $x$.\\

Il est évident que $p$ peut être le polynôme nul.\\

Dans la suite on supposera que $p$ n'est pas $0_{\mathbb{R}[X]}$ et on notera $n$ le degré de $p$.

$$\exists (c, \lambda_1, ..., \lambda_n) \in \mathbb{R}^{n+1}/ p(X)=c(X-\lambda_1)...(X - \lambda_n) \text{ et } c \neq 0$$

On pose : $Q(X) = 16(X - 1)p(X)$\\

On peut calculer le coefficient dominant de $Q$ et on trouve qu'il est égal à $16c$, or, par hypothèse, on a aussi : $$Q(X) = (X - 16)p(2X)$$

Si on recalcule le coefficient dominant de $Q$ à partir de cette nouvelle égalité on obtient $2^nc$, ainsi, on a l'égalité : $$2^n = 16\text{ car } c \neq 0$$

et donc, $n = 4$ ie. $p$ est de degré $4$\\

Ensuite, remarquons que $1$ est une racine de $Q$, or $1$ n'est pas une racine du polynôme $X - 16$ donc c'en est une du polynôme $p(2X)$ et donc $2$ est une racine de $p$. On a alors, une nouvelle racine de Q et réitérant le procédé, on trouve que $4$, $8$ et $16$ sont également des racines de $p$ et $16$, la dernière racine obtenue ainsi, est racine de $X - 16$ donc cette dernière racine ne permet pas d'en obtenir une nouvelle. $p$ étant de degré $4$, on a trouvé toutes ses racines et donc : $$p(X) = c(X-2)(X-4)(X-8)(X-16)$$

Synthèse : Soit p un polynôme de la forme :

$$p(X) = c(X-2)(X-4)(X-8)(X-16) \text{ avec } c \in \mathbb{R}$$

Premièrement, il est évident que $p$ est un polynôme à coefficient réels.\\

Ensuite, si $c$ est nul alors $p$ est le polynôme nul et il est alors évident que $p$ satisfait $(E)$ pour tout réel $x$.\\

Si $c \neq 0$\\

Montrons que : $$\forall x \in \mathbb{R}, (x - 16)p(2x) = 16(x - 1)p(x)$$

$\forall x \in \mathbb{R},$

\begin{equation}
\begin{split}
16(x - 1)p(x) &= 2^{4}(x - 1)c(x-2)(x-4)(x-8)(x-16)\\\
&= c(2x - 2)(2x-4)(2x-8)(2x-16)(x-16)\\
16(x - 1)p(x) &= (x - 16)p(2x)
\end{split}
\end{equation}

Ainsi, $p$ satisfait bien $(E)$ pour tout réel $x$.\\

Conclusion : Les polynômes $p$ de $\mathbb{R}[X]$ satisfaisant $(E)$ pour tout réel $x$ sont les polynômes de la forme : $$p(X) = c(X-2)(X-4)(X-8)(X-16) \text{ avec } c \in \mathbb{R}$$



\end{document}
